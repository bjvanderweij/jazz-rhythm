\chapter{Introduction}
\label{sec:introduction}


%Thesis:
%\begin{itemize}
%\item Many models use a prior proportional to score complexity, we use a more advanced notion of score likelihood
%\item Many models use a likelihood where expressive timing is additive noise. We think there is a correlation between %structure (down- and upbeats) and expressive timing
%\item Some models use tempo curves to compensate errors introduced by tempo changes, we don't need tempo curves, do we?
%\item A structural analysis is more powerful (and simpler?) than Temperley's common practice rhythm models
%\end{itemize}

% Introduction of the field

% Problem specification:
% 	Analysis of rhythms. Not interested in audio recordings, purely interested in structure behind rhythm
% Hypothesis
% 	Subdivision analyses are useful
%	Although expression is complicated, there may be easily generalisable phenomenons in expression that can be linked to downbeats and upbeats.