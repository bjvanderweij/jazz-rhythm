\abstract{%
An approach is presented here that interprets rhythmic structure of monophonic music performances based on onset times. Rhythmic structure is represented as subdivision trees. A chart parsing algorithm that uses probabilistic beam search is presented to construct these structures. A Bayesian probabilistic model is used in which the probability of a performance given a rhythmic structure is modelled by an expression model and the probability of the rhythmic structure itself is modelled by a rhythm model. The system is completely tempo-independent and therefore well-suited for studying expression.

It is suggested that local expressive deviations, such as stretching certain beats to emphasise them, are related to rhythmic structure. An expression-aware model is proposed that is sensitive to these deviations and it is suggested that an expression-aware model may improve parser performance. In addition to that, an alternative expression model is proposed that treats expressive deviation as noise.

Furthermore, a new corpus containing monophonic performances of well-known jazz standards annotated with rhythmic structure is presented here and used to evaluate the parser.

The performance of both our expression models is compared to a baseline. This baseline uses an uninformed rhythm model that ranks every rhythm as equally likely and an expression model that treats expressive deviation as noise. It was found that while our expression-aware model did not perform better than the baseline, the alternative expression model did.
}